\clearpage
\makeatletter
\efloat@restorefloats
\makeatother


\begin{appendix}
\renewcommand{\appendixname}{\bf{Supplementary Material}}
\renewcommand{\figurename}{Supplementary Figure}
\renewcommand{\tablename}{Supplementary Table}
\renewcommand{\thefigure}{S\arabic{figure}} \setcounter{figure}{0}
\renewcommand{\thetable}{S\arabic{table}} \setcounter{table}{0}
\renewcommand{\theequation}{S\arabic{table}} \setcounter{equation}{0}

\hypertarget{section}{%
\section{}\label{section}}

\hypertarget{supplementary-methods}{%
\subsection{Supplementary Methods}\label{supplementary-methods}}

\hypertarget{calculating-global-geographic-and-linguistic-proximity-matrices}{%
\subsubsection{Calculating global geographic and linguistic proximity
matrices}\label{calculating-global-geographic-and-linguistic-proximity-matrices}}

Geographic distance between two nations was calculated as the logged
geodesic distance between country capital cities (data from the R
package \emph{maps}\textsuperscript{1}) using the R package
\emph{geosphere}\textsuperscript{2}. The geographic proximity matrix was
computed as one minus the log geographic distance matrix scaled between
0 and 1.

Linguistic proximity between two nations was calculated as the cultural
proximity between all languages spoken within those nations, weighted by
speaker percentages. We acquired cultural proximity data by combining
the language family trees provided by Glottolog v3.0\textsuperscript{3}
into one global language tree (undated and unresolved). We calculated
cultural proximity \(s\) between two languages \(j\) and \(k\) as the
distance (in number of nodes traversed) of their most recent common
ancestor \(i\) to the root of the tree, through the formula:

\[
s_{jk} = \frac{n_{r}-n_{i}}{n_{r}}
\]

where \(n_{r}\) is the maximum path length (in number of nodes
traversed) leading to the pan-human root \(r\), and \(n_{i}\) is the
maximum path length leading to node \(i\). We then combined these
proximities with speaker data from Ethnologue 21\textsuperscript{4} and
compared every language spoken within those nations by at least 1
permille of the population, weighted by speaker percentages, through the
formula:

\[
w_{lm} = {\Sigma}{\Sigma}p_{lj}p_{mk}s_{jk}
\]

where \(p_{lj}\) is the percentage of the population in nation \(l\)
speaking language \(j\), \(p_{mk}\) is the percentage of the population
in nation \(m\) speaking language \(k\), and \(s_{jk}\) is the proximity
measure between languages \(j\) and \(k\)\textsuperscript{5}. The
resulting linguistic proximity matrix was also scaled between 0 and 1
before analysis.

\hypertarget{bayesian-models-for-reanalysis}{%
\subsubsection{Bayesian models for
reanalysis}\label{bayesian-models-for-reanalysis}}

We provide model formulae for our reanalyses of cross-national
correlations, for a general bivariate case with standardised outcome
\(Y\) and predictor \(X\) variables. In the naive regression model
without controls for non-independence:

\[
\begin{aligned}
Y_{i} &\sim \text{Normal}(\mu_{i},\sigma) \\
\mu_{i} &= \alpha + \beta X_{i} \\
\alpha &\sim \text{Normal}(0, 0.4) \\
\beta &\sim \text{Normal}(0, 0.4) \\
\sigma &\sim \text{Exponential}(5)
\end{aligned}
\] The priors in this model were arrived at by prior predictive checks,
with wider priors making predictions beyond the scale of standardised
outcome variables and narrower priors being too informative.

\newpage

To control for spatial non-independence, we add a Gaussian process to
this model and feed it a scaled geographic distance matrix \(D\) based
on Euclidean distances between latitude and longitude coordinates. This
distance matrix is computed internally by the R package
\emph{brms}\textsuperscript{6}. The Gaussian process uses an
exponentiated quadratic covariance kernel, the only covariance kernel
currently supported by \emph{brms}. The model formula is:

\[
\begin{aligned}
Y_{i} &\sim \text{Normal}(\mu_{i},\sigma) \\
\mu_{i} &= \alpha + \kappa_{\text{NATION}[i]} + \beta X_{i} \\
\begin{pmatrix}
\kappa_{1} \\ \kappa_{2} \\ \kappa_{3} \\ ... \\ \kappa_{n}
\end{pmatrix} &\sim \text{MVNormal}
\begin{pmatrix}
\begin{pmatrix}
0 \\ 0 \\ 0 \\ ... \\ 0
\end{pmatrix},\textbf{K}
\end{pmatrix}\\
\textbf{K}_{ij} &= sdgp^2 \text{exp} \big (-D_{ij}^2 / (2 lscale^2) \big )\\
\alpha &\sim \text{Normal}(0, 0.4) \\
\beta &\sim \text{Normal}(0, 0.4) \\
\sigma &\sim \text{Exponential}(5) \\
sdgp &\sim \text{Exponential}(5) \\
lscale &\sim \text{InverseGamma}(?,?)
\end{aligned}
\] where \(n\) is the number of nations, and \(D^2_{ij}\) reflects the
squared Euclidean distances between latitude and longitude coordinates
for the \(i\)-th and \(j\)-th nations. Notice that the inverse gamma
prior on \(lscale\) is left undetermined. This is because the
\emph{brms} package intelligently tunes the prior for this parameter
based on the covariates of the Gaussian process (see
https://betanalpha.github.io/assets/case\_studies/gp\_part3/part3.html).

\newpage

To control for cultural phylogenetic non-independence, we manually
specify the covariance structure for nation random intercepts using a
pre-computed linguistic proximity matrix \(L\) (see previous section).
The covariance between two nations is assumed to be linearly
proportional to the linguistic proximity between those nations. This
assumption is justified if we assume that cultural traits evolve via
Brownian motion along a language phylogeny. The non-centered
parameterisation of this model is:

\[
\begin{aligned}
Y_{i} &\sim \text{Normal}(\mu_{i},\sigma) \\
\mu_{i} &= \alpha + z_{\text{NATION}[i]}\sigma_{\alpha}L + \beta X_{i} \\
\alpha &\sim \text{Normal}(0, 0.4) \\
\beta &\sim \text{Normal}(0, 0.4) \\
z_{j} &\sim \text{Normal}(0, 1)\\
\sigma_{\alpha} &\sim \text{Exponential}(5) \\
\sigma &\sim \text{Exponential}(5)
\end{aligned}
\] \newpage

Finally, we can control for spatial and cultural phylogenetic
non-independence simultaneously by including both a Gaussian process
over latitude and longitude coordinates \emph{and} nation random
intercepts that covary according to linguistic proximity. The resulting
model is as follows:

\[
\begin{aligned}
Y_{i} &\sim \text{Normal}(\mu_{i},\sigma) \\
\mu_{i} &= \alpha + \kappa_{\text{NATION}[i]} + z_{\text{NATION}[i]}\sigma_{\alpha}L + \beta X_{i} \\
\begin{pmatrix}
\kappa_{1} \\ \kappa_{2} \\ \kappa_{3} \\ ... \\ \kappa_{n}
\end{pmatrix} &\sim \text{MVNormal}
\begin{pmatrix}
\begin{pmatrix}
0 \\ 0 \\ 0 \\ ... \\ 0
\end{pmatrix},\textbf{K}
\end{pmatrix}\\
\textbf{K}_{ij} &= sdgp^2 \text{exp} \big (-D_{ij}^2 / (2 lscale^2) \big )\\
\alpha &\sim \text{Normal}(0, 0.4) \\
\beta &\sim \text{Normal}(0, 0.4) \\
z_{j} &\sim \text{Normal}(0, 1)\\
\sigma_{\alpha} &\sim \text{Exponential}(5) \\
\sigma &\sim \text{Exponential}(5) \\
sdgp &\sim \text{Exponential}(5) \\
lscale &\sim \text{InverseGamma}(?,?)
\end{aligned}
\]

\newpage

\hypertarget{supplementary-results}{%
\subsection{Supplementary Results}\label{supplementary-results}}

\hypertarget{geographic-and-cultural-phylogenetic-signal-estimates-for-human-development-index-and-ingleharts-value-dimensions}{%
\subsubsection{Geographic and cultural phylogenetic signal estimates for
Human Development Index and Inglehart's value
dimensions}\label{geographic-and-cultural-phylogenetic-signal-estimates-for-human-development-index-and-ingleharts-value-dimensions}}

Controlling for shared cultural ancestry, the proportion of
national-level variance explained by spatial proximity was 0.38 for the
Human Development Index (95\% credible interval {[}0.20 0.60{]};
\(\text{BF}_{\neq0}\) \textgreater{} 100), 0.43 for traditional values
(95\% CI {[}0.18 0.72{]}; \(\text{BF}_{\neq0}\) \textgreater{} 100), and
0.18 for survival values (95\% CI {[}0.00 0.44{]}; \(\text{BF}_{\neq0}\)
= 1.45). Controlling for spatial proximity, the proportion of variance
explained by shared cultural ancestry was 0.60 for the Human Development
Index (95\% CI {[}0.39 0.79{]}; \(\text{BF}_{\neq0}\) \textgreater{}
100), 0.55 for traditional values (95\% CI {[}0.26 0.79{]};
\(\text{BF}_{\neq0}\) \textgreater{} 100), and 0.79 for survival values
(95\% CI {[}0.53 0.97{]}; \(\text{BF}_{\neq0}\) \textgreater{} 100).

\newpage

\hypertarget{supplementary-figures}{%
\subsection{Supplementary Figures}\label{supplementary-figures}}

(ref:dagCaption) \emph{A causal directed acyclic graph of spatial and
cultural phylogenetic non-independence in cross-national studies.} We
are interested in estimating the direct effect of national-level
exposure \(X\) on national-level outcome \(Y\). But these variables are
confounded by their common unobserved cause \(U\). \(U\) is a stand-in
for shared environmental, ecological, and geographic causes
(e.g.~climate, biodiversity, physical topography) and cultural and
institutional causes (e.g.~cultural norms, political systems). In this
causal model, we cannot condition on \(U\) directly, as it is
unobserved. However, geographic \(G\) and linguistic \(L\) relationships
between societies influence \(U\), since changing a nation's spatial
distance to or shared cultural ancestry with other nations will change
its environmental and cultural traits. We can thus use \(G\) and \(L\)
to model the covariation between \(X\) and \(Y\) induced by \(U\).
Failing to do this and simply estimating the bivariate correlation
between \(X\) and \(Y\) will produce spurious relationships and
residuals that are spatially and culturally non-independent around the
world (see Figure @ref(fig:spatialExample)).

\begin{figure}
\centering
\includegraphics{manuscript_files/figure-latex/dag-1.pdf}
\caption{(\#fig:dag)(ref:dagCaption)}
\end{figure}

\newpage

(ref:plotSim1Caption) \emph{Distribution of cross-national correlations
from simulation study under moderate spatial autocorrelation.} In these
simulations, the strength of spatial autocorrelation is set to 0.5 for
both outcome and predictor variables. For frequentist regression models,
point ranges represent correlation estimates and 95\% confidence
intervals. For Bayesian Gaussian process regression models, point ranges
represent posterior means and 95\% credible intervals. Correlations are
ordered by effect size independently in each panel. Red point ranges
indicate that the slope is ``significant'' (i.e.~the 95\% confidence /
credible interval excludes zero). Black point ranges indicate that the
slope is ``not significant''.

\begin{figure}
\centering
\includegraphics{manuscript_files/figure-latex/plotSim1-1.pdf}
\caption{(\#fig:plotSim1)(ref:plotSim1Caption)}
\end{figure}

\newpage

(ref:plotReplications2Caption) \emph{Reanalysis models fitted to raw
data, for economic development (a) and cultural values (b) studies.}
Data points are labelled using ISO 3166-1 alpha-2 letter country codes.
In all reanalyses, outcome and predictor variables are standardised,
making regression slopes comparable to Pearson's correlation
coefficients. Lines and shaded areas represent posterior median
regression lines and 95\% credible intervals. For models with covariates
(Adamzyck and Pitt 2009; Gelfand et al.~2011), marginal effects are
presented holding all covariates at zero or their reference categories.

\begin{figure}
\centering
\includegraphics{manuscript_files/figure-latex/plotReplications2-1.pdf}
\caption{(\#fig:plotReplications2)(ref:plotReplications2Caption)}
\end{figure}

\newpage

(ref:plotReplications3Caption) \emph{Posterior estimates of Gaussian
process functions mapping spatial autocorrelation onto geographic
distance from our reanalyses of economic development (a) and cultural
values (b) studies.} Estimates are from models additionally controlling
for cultural phylogenetic non-independence. The y-axis represents the
amount of spatial autocorrelation between data points with increasing
distance between those points on the x-axis (logged distance in
kilometres). Lines and shaded areas represent median posterior spatial
autocorrelation functions and 50\% and 95\% credible intervals.

\begin{figure}
\centering
\includegraphics{manuscript_files/figure-latex/plotReplications3-1.pdf}
\caption{(\#fig:plotReplications3)(ref:plotReplications3Caption)}
\end{figure}

\newpage

(ref:plotReplications4Caption) \emph{Posterior estimates of cultural
phylogenetic signal from our reanalyses.} Estimates are from models
additionally controlling for spatial non-independence. Cultural
phylogenetic signal is operationalised as the proportion of
national-level variance explained by linguistic proximity between
nations. Ridges are full posterior distributions, and point ranges are
posterior medians and 95\% credible intervals.

\begin{figure}
\centering
\includegraphics{manuscript_files/figure-latex/plotReplications4-1.pdf}
\caption{(\#fig:plotReplications4)(ref:plotReplications4Caption)}
\end{figure}

\newpage

\hypertarget{supplementary-references}{%
\subsection*{Supplementary References}\label{supplementary-references}}
\addcontentsline{toc}{subsection}{Supplementary References}

\hypertarget{refs}{}
\leavevmode\hypertarget{ref-Brownrigg2018}{}%
1. Brownrigg, R. \emph{maps: Draw geographical maps}. (2018).

\leavevmode\hypertarget{ref-Hijmans2019}{}%
2. Hijmans, R. J. \emph{geosphere: Spherical trigonometry}. (2019).

\leavevmode\hypertarget{ref-Glottolog}{}%
3. Hammarström, H., Forkel, R., Haspelmath, M. \& Bank, S.
\emph{Glottolog 3.0}. (Max Planck Institute for the Science of Human
History, 2017).
doi:\href{https://doi.org/10.5281/zenodo.4061162}{10.5281/zenodo.4061162}.

\leavevmode\hypertarget{ref-Ethnologue}{}%
4. \emph{Ethnologue: Languages of the world}. (SIL International, 2018).

\leavevmode\hypertarget{ref-Eff2008}{}%
5. Eff, E. A. Weight matrices for cultural proximity: Deriving weights
from a language phylogeny. \emph{Structure and Dynamics} \textbf{3},
(2008).

\leavevmode\hypertarget{ref-Burkner2017}{}%
6. Bürkner, P.-C. brms: An R package for Bayesian multilevel models
using Stan. \emph{Journal of Statistical Software} \textbf{80}, 1--28
(2017).
\end{appendix}
